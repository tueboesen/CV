%-------------------------------------------------------------------------------
%	SECTION TITLE
%-------------------------------------------------------------------------------
\cvsection{Summary}


%-------------------------------------------------------------------------------
%	CONTENT
%-------------------------------------------------------------------------------
%\begin{cventries}

%---------------------------------------------------------
%My main scientific focus areas are currently: machine learning and protein folding.
%More specifically, within machine learning, I have worked extensively with equivariant and graph-based networks, as well as transfer and self-supervised learning.
%The goal of my work in machine learning has always been to fundamentally understand how neural networks work, which I believe is best done by connecting them with well-established areas of mathematics.
%Parts of this work lead to mimetic neural networks, and constrained neural networks, both of which are novel ideas in machine learning that has given me a deeper understanding of neural networks and allows me to more effectively tailor neural network to specific problems.

I am a machine learning scientist and have previously worked as: software developer, researcher, project lead, and technical advisor/consultant on various projects.
I have a solid foundation in: physics, mathematics, data science, and high-performance-computing.
I am familiar with programming best practices and generating production code.
Within machine learning, I have specialised in physics-informed and graph neural networks, but I have experience with most areas of deep learning, and I have applied machine learning in many different fields.


%Below I highlight a few different fields/problems I have used ML on and the ML methods used to solving the problem:
%      \begin{cvitems} % Description(s) of tasks/responsibilities
%        \item {Mechanical systems using symmetry/constraint obeying neural networks.}
%        \item {Oil predictions using semi-supervised clustering.}
%        \item {Generating geological maps from satellite data using GANS, transfer learning and reversible neural networks.}
%        \item {Image classification using spectral clustering and active learning.}
%        \item {Protein folding using NLP models.}
%        %\item {Used PyRosetta to design protein bindings for KRAS.}
%      \end{cvitems}

%---------------------------------------------------------
%\end{cventries}
